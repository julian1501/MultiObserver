\section{Conclusion}
This BEP has implemented three multi-observer variants and compared them based on memory usage. The 2D-CMO and the 3D-CMO store all state estimates individually, where 3D-CMO is much more memory efficient as compared to the 2D-CMO. Unlike the CMOs, the SSMO condenses all observers into a single state. In order to perform the final estimate selection procedure the original states are still required. These are recovered through transformation matrices. Although the shared state of the SSMO requires much less memory than all the individual states of the CMOs, the transformation matrices used to recover the original states still requires significant memory. 

In a direct comparison between the 2D-CMO, 3D-CMO and the SSMO, the 3D-CMO uses the least memory. It should be noted that the 3D-CMOs memory requirements are still significantly too large for an implementation in a system with a large number of outputs.

The MOs extension to observe nonlinear systems remains incomplete, creating aggregate sensors seems to be a promising solution to the issues presented in this BEP. Although the SAMO seems to function well under the tested circumstances, no statements have been made about its implementation on general systems. The LAMO does not show immediate improvements over the SAMO.

Let us note several limitations of this BEP. The size comparison between multiple linear MOs does not consider systems with $P$-observers using a size larger than 1, selecting a different value can be required in order to properly observe a system. This could cause a different total MO size, although the size differences between the MO implementations are believed to be similar. This report has not considered MOs observing a system while feedback control is applied to the system and no effort has been made to select observer eigenvalues based on noise rejection and fast error suppression. Implementing feedback control could require changes to the structure of the CMO in order to select the desired eigenvalues. The Matlab implementations themselves have become overcomplicated in order to run the observers simultaneously.

In order to make a real-world MO implementation possible, more effort needs to be put in reducing the required memory. There is also still a need for further research into the nonlinear extension of the 3D-CMO, the feasibility of the SAMO is questionable regarding the increased size as compared to a linear MO. The unexplained behaviour of the LAMO could be key to expanding the functionality beyond the limit of the attacked aggregate sensors. Although it is also possible that it can never guarantee a secure state estimate beyond this limit.
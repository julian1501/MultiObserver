\section{Problem Statement}
Cyber physical systems (CPSs) are physical systems that are controlled by computer algorithms, where actuator input is based on sensors that measure the state of the physical system. CPSs are often large, geographically dispersed, safety-critical systems that play a crucial role in modern infrastructure, such as power grids or water distribution systems. There are two categories in which most cyberattacks on CPSs can be classified: Denial of Service attakcs (DoS) and deception attacks \cite{Ding2021SecureSurvey}. A DoS attack aims to consume a significant amount of the victims resources: CPU cycles, memory or network bandwidth. The victim is then not able to use these resources for the intended purpose, for CPSs this often means that sensors do not provide any readings \cite{Yu2014AnAttacks}. A deception attack aims to completely take over a sensor, the attacker injects false data into the system \cite{Serpanos2022FalseSystems}. A deception attack is also known as a false data injection (FDI) attack. A system that is under an FDI attack still takes readings from the attacked sensor and makes decisions based on those readings. An FDI attack cannot only cause disruption of service, it can inject false data that aims to maximize long term damage to a system. \\ 

These types of attacks on CPSs can have severe consequences on our (industrial) infrastructure. The Industroyer and Stuxnet attacks serve as infamous examples \cite{Lameiras2022Industroyer:Grid}\cite{Kushner2013TheStuxnet}. Stuxnet first lay dormant recording normal operating conditions, later it would manipulate the sensor data and cause damage to the industrial equipment. During the periods that Stuxnet manipulated sensors, it would replay the recorded data which was sent to the verification systems and were thus unaware of any alarming operating conditions \cite{Fidler2011Was_Stuxnet_an_Act_of_War_Decoding_a_Cyberattack}. There are a number of strategies that can be employed to prevent such an attack, for example: preventing a malicious actor from gaining access to your system. Another approach, the focus of this report, is designing a system that is robust with respect to attacks. \\

The control systems within CPSs often need information about all state variables used to describe the system. Measuring all state variables can be expensive, impractical or even impossible \cite{yappa}. An observer or state estimator can be used to create an estimate of all state variables from the subset of measured state variables. Considering the possible attacks on these systems, it is desirable to have the ability of providing a secure state estimate: a correct state estimate constructed from a set of noisy sensors under attack \cite{Shoukry2017SecureApproach}. This problem is also known as secure state estimation (SSE). There are several approaches to achieve SSE, such as \textcolor{red}{find a few approaches.}

This report will focus on the SSE method presented in \cite{Chong2015ObservabilityAttacks} and \cite{Chong2020AAttacks}, where an observer-based estimator is introduced. Multiple observers are created and observe the system besides each other, such an observer will be referred to as a \textit{conventional multi-observer} (CMO) in this report. This approach has an implementation bottleneck due to the large number of observers required to provide SSE, requiring large amounts of memory. A strategy to reduce memory usage is presented in \cite{Chong2023MemoryAlgorithms}, where a common state is shared between all observers. An observer with this architecture will be referred to as a \textit{state-sharing multi-observer} (SSMO). \\

This BEP aims to implement both a CMO and an SSMO, compare their performance and implement an SSMO in a SSE context. In Chapter \ref{ch:system-definition} a mass-spring-damper system will be introduced. This system will serve as a case study, all MOs that are implemented in this report are applied to this system. Chapter \ref{ch:state-estimation} will introduce some important concepts fundamental principles regarding state estimation that are required to implement MOs. The CMO and SSMO will be introduced in Chapters \ref{ch:cmo} and \ref{ch:ssmo} respectively. Details of the implementation(s) will be discussed in Chapter \ref{ch:matlab-implementation}
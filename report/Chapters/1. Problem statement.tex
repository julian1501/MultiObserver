\section{Problem Statement}
Cyber physical systems (CPSs) are physical systems that are controlled by computer algorithms, actuator input is based on sensors that measure the state of the physical system. CPSs are often large, geographically dispersed, safety-critical systems that play a crucial role in modern infrastructure, such as power grids or water distribution systems. There are two categories in which most cyberattacks on CPSs can be classified: Denial of Service attakcs (DoS) and deception attacks \cite{Ding2021SecureSurvey}. A DoS attack aims to consume a significant amount of the victims resources: CPU cycles, memory bandwith or network bandwidth. The victim is then not able to use these resources for the intended purpose, for CPSs this often means that sensors do not provide any readings \cite{Yu2014AnAttacks}. A deception attack aims to completely take over a sensor, the attacker injects false data into the system \cite{Serpanos2022FalseSystems}, such an attack is also known as a false data injection (FDI) attack. A system that is under an FDI attack still takes readings from the attacked sensor and makes decisions based on those readings. An FDI attack cannot only cause disruption of service, it can inject false data that aims to maximize long-term damage to a system. 

These types of attacks on CPSs can have severe consequences on our (industrial) infrastructure. The Industroyer and Stuxnet attacks serve as infamous examples \cite{Kushner2013TheStuxnet},\cite{Lameiras2022Industroyer:Grid}. Stuxnet first lay dormant recording normal operating conditions, later it would manipulate the sensor data and cause damage to the industrial equipment. During the periods that Stuxnet manipulated sensors, it would replay the recorded data which was sent to the verification systems and were thus unaware of any abnormal operating conditions \cite{Fidler2011Was_Stuxnet_an_Act_of_War_Decoding_a_Cyberattack}. There are a number of strategies that can be employed to prevent such an attack, the obvious one: preventing a malicious actor from gaining access to your system in the first place. Another approach, the focus of this report, is designing a system that is robust with respect to attacks. 

The control systems within CPSs often need information about all state variables used to describe the system. Measuring all state variables can be expensive, impractical or even impossible \cite{Buchi2010StateExamples}. An \textit{observer} or \textit{state estimator} can be used to create an estimate of all state variables from the subset of measured state variables. Considering the previously discussed possibility of attacks on these systems, it is strongly desirable to have the ability of providing a secure state estimate: a correct state estimate constructed from a set of noisy sensors under attack \cite{Shoukry2017SecureApproach}. This problem is also known as \textit{secure state estimation} (SSE). 

This report will focus on the SSE method presented in \cite{Chong2015ObservabilityAttacks} and \cite{Chong2020AAttacks}, where an observer-based estimator is introduced. Multiple observers are created and observe the system in parallel, such an observer will be referred to as a \textit{multi-observer} (MO) in this report. An MO can be implemented by storing each observer separately, this will be referred to as a \textit{conventional multi-observer} (CMO). This approach has an implementation bottleneck due to the large number of observers required to provide SSE, requiring large amounts of memory. A strategy to reduce memory usage is presented in \cite{Chong2023MemoryAlgorithms}, where a common state is shared between all observers. An observer with this architecture will be referred to as a \textit{state-sharing multi-observer} (SSMO). The main aim of this BEP is to provide a comparison between the two multi-observers based on memory usage.

In Chapter \ref{ch:system-definition} a mass-spring-damper system will be introduced, this system will serve as a case study throughout the report. Chapter \ref{ch:state-estimation} discusses some important concepts regarding state estimation, such as observability and a single state estimator. These concepts underlie the MOs that are in the rest of the report. Chapter \ref{ch:cmo} is the first chapter that discusses the CMO, the state estimates are constructed and, more importantly, the selection procedure that governs selecting the final state estimate is laid out. Two variants of a CMO are discussed, both MO architectures are explained and a size estimation is presented. Chapter \ref{ch:ssmo} discusses the SSMO and its architecture and compares the CMOs to the SSMO in terms of memory usage. In the final chapter, Chapter \ref{ch:nonlinear-mos}, the nonlinear extension to the MO will be investigated. It is not an exhaustive discussion of nonlinear MOs, but serves as a brief look at what happens when a standard nonlinearity is applied on an MO.
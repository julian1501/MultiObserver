\section{State estimation}\label{ch:state-estimation}
This chapter first discussed the motivation for full state estimation. It also introduces the concept of observability and how it relates to state estimation. A standard 'single-observer' will be derived for linear systems and it will be expanded to also observe system with a non-linearity.


\subsection{Motivation for full state estimation}
Let us first discuss why there is a need for state estimation, consider the CLTI system
\begin{equation}\label{eqn:standard-noiseless-system}
    \dot{x} = Ax + Bu, \quad y = Cx + Du.
\end{equation}
The solution to this system can be written as
\begin{equation*}
    \begin{split}
        x &= e^{tA}x_{0} + \int_{0}^{t}e^{(t-\tau)A}Bu(\tau)d\tau \\
        y &= Ce^{tA}x_{0} + \int_{0}^{t}Ce^{(t-\tau)A}Bu(\tau)d\tau + Du.
    \end{split}
\end{equation*}
where $x_0 = x(0)$ \cite[Eqn. 6.4]{Hespanha2018LinearTheory}. When there is no input, i.e. $u=0$, system \eqref{eqn:standard-noiseless-system} reduces to
\begin{equation*}
    \dot{x} = Ax, \quad y = Cx
\end{equation*}
which has the solution
\begin{equation}\label{eqn:zero-input-solution}
    x = e^{tA}x_0, \quad y = Ce^{tA}x_0.
\end{equation}
Let us now consider the definition of Lyapunov stability as in \cite[Th. 8.2]{Hespanha2018LinearTheory}, which states that a system is stable if and only if all the eigenvalues $\lambda_i,i=1,2,\dots,n$ of $A$ have strictly negative real parts. The matrix exponential in equation \eqref{eqn:zero-input-solution} computed by writing the matrix $A$ into its Jordan normal form $J=PAP^{-1}$, the solution can then be written as
\begin{equation}\label{eqn:jordan-form-exponential}
    x = P^{-1}e^{tJ}Px_0
\end{equation}
\cite[Section 7.3]{Hespanha2018LinearTheory}. When all eigenvalues of $A$ have an algebraic multiplicity of one (no duplicate eigenvalues), the matrix $J$ is of the form
\begin{equation}\label{eqn:diagonal-matrix-exponential}
    J =
    \begin{bmatrix}
        \lambda_1 & 0 & \cdots & 0 \\
        0 & \lambda_2 & \cdots & 0 \\
        \vdots & \vdots & \ddots & \vdots \\
        0 & 0 & \cdots & \lambda_n \\
    \end{bmatrix} \implies 
    e^{tJ} = 
    \begin{bmatrix}
        e^{t\lambda_1} & 0 & \cdots & 0 \\
        0 & e^{t\lambda_2} & \cdots & 0 \\
        \vdots & \vdots & \ddots & \vdots \\
        0 & 0 & \cdots & e^{t\lambda_n} \\
    \end{bmatrix}
\end{equation}
\cite[Section 7.1]{Hespanha2018LinearTheory}. Substituting equation \eqref{eqn:diagonal-matrix-exponential} into equation \eqref{eqn:jordan-form-exponential} shows that if all eigenvalues of $A$ have strictly negative real parts, $x \rightarrow 0$ as $t \rightarrow \infty$. If the matrix $A$ is not stable, but stability properties are still desired. The control law
\begin{equation}\label{eqn:feedback-control-law}
    u = Kx \implies x = Ax + BKx \implies \dot{x} = (A+BK)x
\end{equation}
can be used to stabilize the system, if and only if the system \eqref{eqn:standard-noiseless-system} is stabilizable \cite[Th. 14.5]{Hespanha2018LinearTheory}. As can be seen in the equation, the control law \eqref{eqn:feedback-control-law} requires knowing the full state $x$, which can be very costly or not possible to measure. So we will now work towards creating a state estimate $\hat{x} \rightarrow x$ as $t \rightarrow \infty$.
 
\subsection{Observability}
We now aim to reconstruct the full state $x$ from the output $y$ and the input $u$. The following derivation is based on \cite{StephenBoyd2009ObservabilitySlides}, consider system \eqref{eqn:standard-noiseless-system} where $x$ is unknown and $y$ and $u$ are known. The derivatives of $y$ 
\[
\begin{split}
y &= Cx + Du \\
\dot{y} &= C\dot{x} + Du =  CAx + CBu + D\dot{u} \\
\Ddot{y} &= CA\dot{x} + CBu + Du = CA^{2}x + CABu + CB\dot{u} + D\Ddot{u} \\
\vdots \\
y^{(n)} &= CA^{n}x + CA^{n-1}Bu + CA^{n-2}B\dot{u} + \dots + CABu^{(n-2)} + CBu^{(n-1)} + Du^{(n)}
\end{split}
\]
will be used to reconstruct $x$ from $y$ and $u$. These derivatives can be combined into
\[ \mathbf{y} =  \mathcal{O}x + \mathcal{K}\mathbf{u}
,\]
where
\[\mathbf{y}=
\begin{bmatrix}
    y \\
    y^{(1)} \\
    y^{(2)} \\
    \vdots \\
    y^{(n-1)} \\
\end{bmatrix} \quad \text{and} \quad
\mathbf{u} = 
\begin{bmatrix}
    u \\
    u^{(1)} \\
    u^{(2)} \\
    \vdots \\
    u^{(n-1)} \\
\end{bmatrix},
\]
are vectors containing the derivatives of the output and the input,
\begin{equation}\label{eqn:observability-matrix}
    \mathcal{O}=
    \begin{bmatrix}
        C \\
        CA \\
        CA^2 \\
        \vdots \\
        CA^{n-1}
    \end{bmatrix}    
\end{equation}

is the \textit{observability matrix} and
\[ \mathcal{K}=
\begin{bmatrix}
    D & 0 & 0 & \hdots & 0\\
    CB & D & 0 & \hdots & 0\\
    CAB & CB & D & \hdots & 0 \\
    \vdots & \vdots & \vdots & \ddots & \vdots \\
    CA^{n-2}B & CA^{n-3}B & CA^{n-4}B & \hdots & D
\end{bmatrix}.
\]
Isolating x results in
\begin{equation}\label{eqn:state-estimation-w-observability-matrix}
    x = \mathcal{O}^{-1}(\mathbf{y}-\mathcal{K}\mathbf{u}).
\end{equation}
As can be seen in \eqref{eqn:state-estimation-w-observability-matrix} observability matrix $\mathcal{O}$ is required to be invertible to reconstruct $x$ from $y$ and $u$. If a system's $A$ and $C$ matrix fulfil this requirement, the system can be described as \textit{observable}. The invertibility requirement is equivalent to the statement that the matrix $\mathcal{O}$ needs to be full rank \cite[Section 2.9]{Lay2016LinearApplications}, this theorem that can also be found in \cite[Corollary 3.8]{Antsaklis2006LinearProcessing}. The implication of this is that it is only possible to reconstruct the state $x$ if the pair $A,C$ is observable. \\
\subsection{Linear single observer}
Let us now reconstruct the full state of system \eqref{eqn:standard-noiseless-system} from  $y$ and $u$ by defining the state estimate as in \cite[Section 16.5]{Hespanha2018LinearTheory}
\begin{equation}\label{eqn:stable-simple-state-estimator}
    \dot{\hat{x}} = A\hat{x} + Bu, \quad \hat{y} = C\hat{x} + Du.
\end{equation}
We now define the \textit{state estimation error}
\begin{equation}\label{eqn:estimate-error}
    e = \hat{x} - x
\end{equation}
which we differentiate and substitute equations \eqref{eqn:standard-noiseless-system} and \eqref{eqn:stable-simple-state-estimator} into to give
\begin{equation*}
    \dot{e} = \dot{\hat{x}} - \dot{x} = A\hat{x} + Bu - Ax - Bu = Ae.
\end{equation*}
We can now conclude that $e \rightarrow 0$ as $t \rightarrow \infty$ if the matrix $A$ is a stability matrix. Let us now define a state estimator that provides an asymptotically correct state estimate even when $A$ is not a stability matrix as in \cite[Section 16.5]{Hespanha2018LinearTheory}:
\begin{equation}\label{eqn:unstable-simple-state-estimator}
    \dot{\hat{x}} = A\hat{x} + Bu + L(\hat{y} - y), \quad \hat{y} = C\hat{x} + Du.
\end{equation}
We now perform the same analysis on the derivative of the state estimation error
\begin{equation}\label{eqn:error-linear-observer}
    \dot{e} = A\hat{x} + Bu + L(C\hat{x} + Du - Cx - Du) - Ax - Bu = (A+LC)e
\end{equation}
which is similar to the solution in \eqref{eqn:zero-input-solution}. From which we can conclude that if $A+LC$ is a stability matrix $e \rightarrow 0$ as $t \rightarrow \infty$.

\subsection{Nonlinear observer}
Let us now extend the observer \eqref{eqn:unstable-simple-state-estimator} to also correctly estimate systems with a nonlinear contribution $\phi(x)$. Consider the system
\begin{equation*}\label{eqn:nonlinear-system}
    \dot{x} = Ax + Bu + E\phi(y), \quad y = Cx + Du.
\end{equation*}
The observer will be constructed as
\begin{equation}
    \dot{\hat{x}} = A\hat{x} + Bu + E\phi(y) + L(\hat{y} - y), \quad \hat{y} = C\hat{x} + Du.
\end{equation}
Note that the input to the nonlinearity is $y$. Let us now substitute these definitions into the derivative of the error $e=\hat{x}-x$
\begin{equation*}\label{eqn:errror-nonlinear-observer}
    \begin{split}
        \dot{e} = \dot{\hat{x}} - \dot{x} &= A\hat{x} + Bu + E\phi(y) + L(\hat{y} - y) - Ax - Bu - E\phi(y) \\
        &= (A+LC)e + E(\phi(y) - \phi(y)) \\
        &= (A+LC)e.
    \end{split}
\end{equation*}
It is now evident why $\phi(y)$ is used in equation \eqref{eqn:nonlinear-system}, otherwise $e \rightarrow 0$ as $t \rightarrow \infty$ cannot be guaranteed. Because $\phi(y) - \phi(x)$ does not necessarily equal $0$. This does set certain requirements on $y$, variables that affect the nonlinearity need to be measured directly. For the multi mass-spring-damper system as in Chapter \ref{ch:system-definition} this means that all positions $x_a,a=1,2,\dots,b$ need to be measured directly. \\

\subsection{Eigenvalue placement}
Since we have full control over $L$, we will now discuss methods to place eigenvalues of $A+LC$ at desired locations and what constraints are limiting our control over these eigenvalues.

\begin{theorem}[Observer eigenvalues]
\label{th:arbitrary-alc-eigenvalues}
    If the pair $(A,C)$\eqref{eqn:standard-noiseless-system} is observable there exists an $L \in \mathbb{R}^{n \times m}$ that influences all eigenvalues of $A+LC$.
\end{theorem}
\begin{proof}
    This proof is based on \cite[Section 4.2]{Antsaklis2006LinearProcessing}. Suppose that the pair $(A,C)$ is not fully observable and that all eigenvalues of $A+LC$  have been influenced by $L$, it will be shown that this leads to a contradiction. There exists a similarity transformation that separates the observable from the unobservable part: the observable decomposition \cite[Section 16.1]{Hespanha2018LinearTheory}. Performing this transformation on the pair $(A,C)$ leads to
    \begin{equation}
    \begin{split}
        \begin{bmatrix}
            \dot{x}_o \\
            \dot{x}_u
        \end{bmatrix}
        &=
        \begin{bmatrix}
            A_o & 0 \\
            A_{21} & A_u
        \end{bmatrix}
        \begin{bmatrix}
            x_o \\
            x_u
        \end{bmatrix}
        + 
        \begin{bmatrix}
            L_1 \\
            L_2
        \end{bmatrix}
        \begin{bmatrix}
            C_o & 0
        \end{bmatrix} \\
        &=
        \begin{bmatrix}
            A_o + L_1 C_o & 0 \\
            A_{21} + L_2 C_o & A_u
        \end{bmatrix}
    \end{split}
    \end{equation}  
where the pair $(A_o,C_o)$ is observable. A similarity transformation leaves the eigenvalues unchanged \cite[Section 5.2]{Lay2016LinearApplications}. The lower triangular structure of the matrix implies that the eigenvalues of $A_u$ remain the same, which leads to a contradiction: not all the eigenvalues of $A+LC$ have been influenced. Thus, if the pair $(A,C)$ is observable there exists an $L$ that can influence all eigenvalues of $A+LC$.
\end{proof}

An analytical method to place the eigenvalues at arbitrary locations in a MIMO system is presented in \cite[Section 4.2 B]{Antsaklis2006LinearProcessing}, where the system is first transformed into \textit{observer form} by a similarity transformation. After this transformation the deriving the matrix $L$ is convenient. Another option is numerically deriving $L$. Matlab provides the function \texttt{place}, based on the algorithm presented in \cite{Kautsky1985KautskyNV85}.

\textcolor{red}{Unsatisfying conclusion, would prefer to show an actual strategy to choose the eigenvalues.}



\section{State-sharing multi-observer}\label{ch:ssmo}
In this chapter the state-sharing multi-observer (SSMO) will be discussed, aiming to reduce the required memory to store the CMO. The SSMO described in this chapter will provide the estimates as described in \autoref{subsec:state-estimates} and will employ the same selection procedure as described in \autoref{subsec:estimate-selection}.

\subsection{Constructing the state-estimates}
Let
\begin{equation}\label{eqn:ssmo-observer}
    \begin{split}
        \dot{\hat{x}}_o &= \A_o\hat{x}_o - L_oC_ox + Bu -L_o(v_o + \tau_o), \quad \A_o=A+L_{o}C_{o} \\
    \end{split}
\end{equation}
be a set of observers that contains all $J$- and $P$-observers \eqref{eqn:cmo-single-J-observer}\eqref{eqn:cmo-single-P-observer}, such that $\mcO=\mcJ\bigcup\mcP$ and $o=1,2,\dots,|\mcO|$. All $L_o$ must be selected in such a way that all $\A_o$ share the same characteristic polynomial
\begin{equation}\label{eqn:ssmo-char-poly}
    \det(sI-\A) = p(s) = s^n + q_1s^{n-1} + \dots + q_{n-1}s + q_n
\end{equation}
and thus all have equal eigenvalues. Let us now define a matrix $\Tilde{L}_{o}$ such that $\Tilde{L}_oy=L_oy_o$. This can be achieved by padding $L_o$ with zero vectors $z \in \mathbb{R}^{n \times 1}$ \cite{Chong2023MemoryAlgorithms}. Let us now rewrite the observer \eqref{eqn:ssmo-observer} into the following form
\begin{equation}\label{eqn:ssmo-standard-system-form}
    \dot{\hat{x}}_o = \A_o\hat{x}_o + \B_o\eta_o, \quad
    \B_o =
    \begin{bmatrix}
        E & B & -\Tilde{L}_o \\
    \end{bmatrix}, \quad \eta_o =
    \begin{bmatrix}
        \phi(y) \\
        u \\
        y_o + v_o + \tau_o
    \end{bmatrix}.
\end{equation}

Let us now derive transformation matrices $T_o$ that transform all $\A_o=A+L_{o}C_{o}$ into controllable canonical form as in \cite[Sec. 4.3.2]{Hespanha2018LinearTheory}
\begin{equation}\label{eqn:controllable-canonical-form}
    \mathbf{A} =
    \begin{bmatrix}
        -q_1I_l & -q_2I_l & \cdots & -q_{n-1}I_l & -q_nI_l \\
        I_l & 0_l & \cdots & 0_l & 0_l \\
        0_l & I_l & \cdots & 0_l & 0_l \\
        \vdots & \vdots & \ddots & \vdots & \vdots \\
        0_l & 0_l & \cdots & I_l & 0_l \\
    \end{bmatrix}, \quad
    \mathbf{B} = 
    \begin{bmatrix}
        I_l \\ 0_l \\ \vdots \\ 0_l \\ 0_l \\
    \end{bmatrix}
\end{equation}
where $l=N$. The transformation matrices for each observer are $T_o=R_pR_q$,
\begin{equation}
    \begin{split}
         R_p &=
        \begin{bmatrix}
            \B_o & \A_o\B_o & \A^{2}_o\B_o & \cdots & \A^{n-1}_o\B_o \\
        \end{bmatrix} \\
        R_q &=
        \begin{bmatrix}
            I_l & q_1I_l & q_2I_l & \cdots & q_{n-1}I_l \\
            0_l & I_l & q_1I_l & \cdots & q_{n-2}I_l \\
            \vdots & \ddots & \ddots & \ddots & \vdots \\
            0_l & \cdots & 0_l & I_l & q_1I_l \\
            0_l & \cdots & 0_l & 0_l & I_l \\
        \end{bmatrix}.
    \end{split}
\end{equation}
Let us now show that the following
\begin{equation}\label{eqn:A-transformation}
    \begin{split}
        T_o\mathbf{A} &= \A_oT_o \\
        R_pR_q\mathbf{A} &= \A_oR_pR_q \\
    \end{split}
\end{equation}
holds. Let us start by expanding
\begin{equation*}
    \begin{split}
        R_q\mathbf{A} &=  
        \begin{bmatrix}
            I_l & q_1I_l & q_2I_l & \cdots & q_{n-1}I_l \\
            0_l & I_l & q_1I_l & \cdots & q_{n-2}I_l \\
            \vdots & \ddots & \ddots & \ddots & \vdots \\
            0_l & \cdots & 0_l & I_l & q_1I_l \\
            0_l & \cdots & 0_l & 0_l & I_l \\
        \end{bmatrix}
        \begin{bmatrix}
            -q_1I_l & -q_2I_l & \cdots & -q_{n-1}I_l & -q_nI_l \\
            I_l & 0_l & \cdots & 0_l & 0_l \\
            0_l & I_l & \cdots & 0_l & 0_l \\
            \vdots & \vdots & \ddots & \vdots & \vdots \\
            0_l & 0_l & \cdots & I_l & 0_l \\
        \end{bmatrix} \\
        &= 
        \begin{bmatrix}
            0_l & 0_l & 0_l & \cdots & 0_l & 0_l & 0_l \\
            I_l & q_1I_l & q_2I_l & \cdots & q_{n-3}I_l & q_{n-2}I_l & 0_l \\ 
            0_l & I_l & q_1I_l & \cdots & q_{n-4}I_l & q_{n-3}I_l & 0_l \\ 
            \vdots & \vdots & \vdots & \ddots & \vdots & \vdots & \vdots \\
            0_l & 0_l & 0_l & \cdots & q_1I_l & q_2I_l & 0_l \\
            0_l & 0_l & 0_l & \cdots & I_l & q_1I_l & 0_l \\
            0_l & 0_l & 0_l & \cdots & 0_l & I_l & 0_l \\
        \end{bmatrix}.
    \end{split} 
\end{equation*}
We now premultiply this by $R_o$
\begin{equation*}
    \begin{split}
        R_pR_q\mathbf{A} &= 
        \begin{bmatrix}
            \B_o & \A_o\B_o & \A^{2}_o\B_o & \cdots & \A^{n-1}_o\B_o \\
        \end{bmatrix}
        \begin{bmatrix}
            0_l & 0_l & 0_l & \cdots & 0_l & 0_l \\
            I_l & q_1I_l & q_2I_l & \cdots & q_{n-2}I_l & 0_l \\ 
            0_l & I_l & q_1I_l & \cdots & q_{n-3}I_l & 0_l \\ 
            \vdots & \ddots & \ddots & \ddots & \vdots & \vdots \\
            0_l & 0_l & 0_l & \ddots & q_1I_l & 0_l \\
            0_l & 0_l & 0_l & \cdots & I_l & 0_l \\
        \end{bmatrix} \\
        &= 
        \begin{bmatrix}
            \A_o\B_o \\ q_1\A_o\B_o + \A^2_o\B_o \\ q_2\A_o\B_o + q_1\A^2_o\B_o + \A^3_o\B_o \\ \cdots \\ q_{n-2}\A_o\B_o + q_{n-3}\A^2_o\B_o + \cdots + q_1\A^{n-2}_o\B_o + \A^{n-1}_o\B_o \\ 0 \\      
        \end{bmatrix}^T
    \end{split}
\end{equation*}
Where by the Cayley-Hamilton theorem we can rewrite penultimate column as
\begin{equation*}
    \begin{bmatrix}
            \A_o\B_o \\ q_1\A_o\B_o + \A^2_o\B_o \\ q_2\A_o\B_o + q_1\A^2_o\B_o + \A^3_o\B_o \\ \vdots \\ -\A^n_o\B_o \\ 0 \\      
        \end{bmatrix}^T
\end{equation*}

We now expand
\begin{equation*}
    \begin{split}
        \A_oR_pR_q &= 
        \begin{bmatrix}
            \A_o\B_o & \A^2_o\B_o & \A^{3}_o\B_o & \cdots & \A^{n}_o\B_o \\
        \end{bmatrix}
        \begin{bmatrix}
            I_l & q_1I_l & q_2I_l & \cdots & q_{n-1}I_l \\
            0_l & I_l & q_1I_l & \cdots & q_{n-2}I_l \\
            \vdots & \ddots & \ddots & \ddots & \vdots \\
            0_l & \cdots & 0_l & I_l & q_1I_l \\
            0_l & \cdots & 0_l & 0_l & I_l \\
        \end{bmatrix} \\
        &=
        \begin{bmatrix}
            \A_o\B_o \\ 
            q_1\A_o\B_o + \A^2_o\B_o \\ 
            q_2\A_o\B_o + q_1\A^2\B_o + \A^3_o\B_o \\ \vdots \\ 
            q_{n-1}\A_o\B_o + q_{n-2}\A^2_o\B_o + \cdots + \A^{n-1}_o\B_o \\
            q_{n-1}\A_o\B_o + q_{n-2}\A^2_o\B_o + \cdots + \A^{n-1}_o\B_o + \A^n_o\B_o \\
        \end{bmatrix}^T,
    \end{split}
\end{equation*}
where the bottom two rows can be simplified by using the Cayley-Hamilton theorem
\begin{equation*}
    \begin{split}
        \A_oR_pR_q &= 
        \begin{bmatrix}
            \A_o\B_o \\ 
            q_1\A_o\B_o + \A^2_o\B_o \\ 
            q_2\A_o\B_o + q_1\A^2\B_o + \A^3_o\B_o \\ \vdots \\ 
            \A^{n}_o\B_o \\
            0 \\
        \end{bmatrix}^T,
    \end{split}
\end{equation*}
which is equal to $R_oR_q\A_o$. We can now conclude that the matrix $T_o=R_oR_q$ satisfies \eqref{eqn:A-transformation}. Now we will show the same for
\begin{equation}\label{eqn:B-transformation}
    \begin{split}
        T_o\mathbf{B} &= \B_o \\
        R_pR_q\mathbf{B} &= \B_o.
    \end{split}
\end{equation}
Let us expand
\begin{equation*}
    \begin{split}
        R_pR_q\mathbf{B} &=
        \begin{bmatrix}
            \B_o & \A_o\B_o & \A^{2}_o\B_o & \cdots & \A^{n-1}_o\B_o \\
        \end{bmatrix}
        \begin{bmatrix}
            I_l & q_1I_l & q_2I_l & \cdots & q_{n-1}I_l \\
            0_l & I_l & q_1I_l & \cdots & q_{n-2}I_l \\
            \vdots & \ddots & \ddots & \ddots & \vdots \\
            0_l & \cdots & 0_l & I_l & q_1I_l \\
            0_l & \cdots & 0_l & 0_l & I_l \\
        \end{bmatrix}
        \begin{bmatrix}
            I_l \\ 0_l \\ \vdots \\ 0_l \\ 0_l \\
        \end{bmatrix} \\
        &=
        \begin{bmatrix}
            \B_o & \A_o\B_o & \A^{2}_o\B_o & \cdots & \A^{n-1}_o\B_o \\
        \end{bmatrix}
        \begin{bmatrix}
            I_l \\ 0_l \\ \vdots \\ 0_l \\ 0_l \\
        \end{bmatrix} = \B_o \\
    \end{split}
\end{equation*}
which shows that \eqref{eqn:B-transformation} holds. It should be noted that $\mathbf{A}$ and $\mathbf{B}$ are independent of $o$ and since all observers \eqref{eqn:ssmo-observer} share the same characteristic polynomial \eqref{eqn:ssmo-char-poly}, $\mathbf{A}$ and $\mathbf{B}$ are the same for all observers. This means that only one copy of the matrices needs to be stored and all state estimates $\hat{x}$ from the transformation matrices.

\subsection{Observer architecture}
Let us set the initial conditions of all state estimates to be equal, we will choose $0$ for simplicity. When all initial conditions are zero $z$ will also be the same for all state estimates, 
\begin{equation*}
    \dot{z} = \mathbf{A}z + \mathbf{B}\eta, \quad \eta = 
    \begin{bmatrix}
        u \\ y + v + \tau
    \end{bmatrix}.
\end{equation*}
We then transform $z$ into the original state estimates by

\begin{equation*}
    \tilde{x}_{SSMO} = \texttt{pm}(T,z,3),
\end{equation*}
where
\begin{center}
    % \begin{minipage}[t]{0.4\textwidth}
    %     \centering
    %     % first tikzpicture
    %     \begin{equation*}
    %         \begin{tikzpicture}[every node/.style={anchor=north east,fill=white,minimum width=1.2cm,minimum height=7mm}]
            
    %         % Define the displacement as a coordinate
    %         \coordinate (displacement) at (0.9,0.2);
        
    %         \matrix (mLP) [draw,matrix of math nodes]
    %             {
    %             \hat{x}_{\cP}^\mcJ \\
    %             };
        
    %         \matrix (dots2) [draw,matrix of math nodes] at ($(mLP.south west)+(displacement)$)
    %             {
    %             \dots \\
    %             };
        
    %         \matrix (mLp) [draw,matrix of math nodes] at ($(dots2.south west)+(displacement)$)
    %             {
    %             \hat{x}_{1}^\mcP \\
    %             };
        
    %         \matrix (mLJ) [draw,matrix of math nodes] at ($(mLp.south west)+(displacement)$)
    %             {
    %             \hat{x}_{\cJ}^\mcJ \\
    %             };
        
    %         \matrix (dots1) [draw,matrix of math nodes] at ($(mLJ.south west)+(displacement)$)
    %             {
    %             \dots \\
    %             };
        
    %         \matrix (mLj) [draw,matrix of math nodes] at ($(dots1.south west)+(displacement)$)
    %             {
    %             \hat{x}_1^\mcJ \\
    %             };
            
    %         \draw[dashed](mLj.north east)--(mLP.north east);
    %         \draw[dashed](mLj.north west)--(mLP.north west);
    %         \draw[dashed](mLj.south east)--(mLP.south east);
            
    %         \node at ($(-4,-1.8)$) {$\hat{x}_{SSMO}=$};
            
    %         \end{tikzpicture}
    %     \end{equation*}
    % \end{minipage}
    \begin{minipage}[t]{0.4\textwidth}
    \centering
    % Second tikzpicture
        \begin{equation}
            \begin{tikzpicture}[every node/.style={anchor=north east,fill=white,minimum width=2cm,minimum height=7mm}]
            
            % Define the displacement as a coordinate
            \coordinate (displacement) at (1.7,0.2);
        
            \matrix (mLP) [draw,matrix of math nodes]
                {
                T^\mcP_{|\mcP|} \\
                };
        
            \matrix (dots2) [draw,matrix of math nodes] at ($(mLP.south west)+(displacement)$)
                {
                \dots \\
                };
        
            \matrix (mLp) [draw,matrix of math nodes] at ($(dots2.south west)+(displacement)$)
                {
                T^\mcP_{1} \\
                };
        
            \matrix (mLJ) [draw,matrix of math nodes] at ($(mLp.south west)+(displacement)$)
                {
                T^\mcJ_{|\mcJ|} \\
                };
        
            \matrix (dots1) [draw,matrix of math nodes] at ($(mLJ.south west)+(displacement)$)
                {
                \dots \\
                };
        
            \matrix (mLj) [draw,matrix of math nodes] at ($(dots1.south west)+(displacement)$)
                {
                T^\mcJ_1 \\
                };

            \matrix (m0) [draw,matrix of math nodes] at ($(mLj.south west)+(displacement)$)
                {
                0 \\
                };
            
            \draw[dashed](m0.north east)--(mLP.north east);
            \draw[dashed](m0.north west)--(mLP.north west);
            \draw[dashed](m0.south east)--(mLP.south east);
            
            \node at ($(-5,-2.4)$) {$T=$};
            
            \end{tikzpicture}
        \end{equation}
    \end{minipage}
\end{center}
The structure of the storage matrix $\tilde{x}_{SSMO}$ is the same as $\tilde{x}_{3D}$ in \eqref{eqn:A-tilde-3D}. The selection of the final estimate follows the same procedure as described in Section \ref{subsec:estimate-selection}.

\begin{table}[h]
    \centering
    \begin{tabular}{c|c|c}
       Matrix  & Dimensions & Number of elements \\ \hline
       $\Tilde{x}_{SSMO}$  & $ n_x \times 1 \times N_S$ & $_xN_S$ \\
       $\mathbf{A}$ & $n_xN_O \times n_xN_O$ & $n_x^2N_O^2$ \\ 
       $\mathbf{B}$ & $n_xN_O \times n_x$ & $n_x^2N_O$ \\
       $T$ & $n_x \times n_xN_O \times N_S$ & $n_x^2N_ON_S$ \\
    \end{tabular}
    \caption{SSMO system matrix dimensions}
    \label{tab:SSMO-dimensions}
\end{table}

\subsection{Size comparison}
Let us now compare the sizes of the 2D-CMO, 3D-CMO and SSMO as presented in Tables \ref{tab:2D-CMO-dimensions}, \ref{tab:3D-CMO-dimensions} and \ref{tab:SSMO-dimensions} respectively. First, the total number of elements for each MO is derived from the matrices that are needed to construct it. 
\begin{table}[h]
    \centering
    \begin{tabular}{c|c}
        Multi-Observer & Total number of Elements \\
        \hline
        2D-CMO & $n_x^2N_S^2 + N_S(n_x+2n_u)$ \\
        3D-CMO & $n_xN_s(1+2n_x)$ \\
        SSMO   & $n_xN_S(1+n_xN_O) + n_x^2N_O(N_O+1)$ \\
    \end{tabular}
    \caption{Total number of elements in each MO implementation}
    \label{tab:my_label}
\end{table}

We now use the for $N_S$ as in equation \eqref{eqn:NS-approximation} to compare number of Gigabytes required to store the MO. The comparison is made for different combinations of the number of state variables $n_x$ and number of total system outputs $N_O$. Both of these values are ranged from $1$ to $100$ and the storage size for each combination is compared.

\begin{figure}
    \centering
    \includegraphics[width=0.5\linewidth]{}
    \caption{Caption}
    \label{fig:enter-label}
\end{figure}